\documentclass[11pt,a4paper]{article}
\usepackage[utf8]{inputenc}
\usepackage[english]{babel}
\usepackage{amsmath}
\usepackage{amsfonts}
\usepackage{amssymb}
\usepackage{mathtools}
\usepackage{graphicx}
\usepackage{subcaption}

\graphicspath{{/home/jfadanni/Multitask/RNN_ID/Figures/}{/home/jfadanni/Documents/Paper_ID_methods/FCI_graphical_explanation}}
\author{Jacopo Fadanni}
\begin{document}

\begin{figure}
 % FCI graphical explanation and model dataset
 \centering
 \begin{subfigure}[b]{0.32\textwidth}
   \centering
   \includegraphics[width=\textwidth]{FCI/FCI_dataset.pdf}
   \subcaption{}\label{fig:FCIexpl:FCI_dataset}
 \end{subfigure}
 \begin{subfigure}[b]{0.32\textwidth}
   \centering
   \includegraphics[width=\textwidth]{FCI/FCI_dataset_processed.pdf}
   \subcaption{}\label{fig:FCIexpl:FCI_dataset_processed}
 \end{subfigure}
 \begin{subfigure}[b]{0.32\textwidth}
    \centering
    \includegraphics[width=\textwidth]{FCI/FCI_dataset_processed_circles.pdf}
    \subcaption{}\label{fig:FCIexpl:FCI_dataset_processed_circles}
  \end{subfigure}\\ 
  \begin{subfigure}[b]{0.36\textwidth}
    \centering
    \includegraphics[width=\textwidth]{FCI/FCI_fitting_example.pdf}
    \subcaption{}\label{fig:FCIexpl:FCI_fitting_example}
  \end{subfigure}
  \begin{subfigure}[b]{0.30\textwidth}
    \centering
    \includegraphics[width=\textwidth]{Model_dataset/plane_3d.pdf}
    \subcaption{}\label{fig:FCIexpl:Model_dataset_plane_3d}
  \end{subfigure}
  \begin{subfigure}[b]{0.30\textwidth}
    \centering
    \includegraphics[width=\textwidth]{Model_dataset/swissroll_3d.pdf}
    \subcaption{}\label{fig:FCIexpl:Model_dataset_swissroll_3d}
  \end{subfigure}
 \caption{The Full Correlation Integral (FCI) method graphically explained.
 (\subref{fig:FCIexpl:FCI_dataset}) an example dataset;
 (\subref{fig:FCIexpl:FCI_dataset_processed}) The first step of FCI is the centering and normalization of the dataset. In this step the datapoints are projected on an unitary sphere in $D-1$ dimensions;
 (\subref{fig:FCIexpl:FCI_dataset_processed_circles}) Second step: the correlation integral of the dataset is computed for different radii;
 (\subref{fig:FCIexpl:FCI_fitting_example}) ID estimation: given $r/r_0$ and $\rho({r/r}_0)$, using the equation %% eq FCI
 the ID is computed. To obtain the true ID increase the estimated ID by one;
 (\subref{fig:FCIexpl:Model_dataset_plane_3d}) A dataset with uniform distribution on a plane, $ID=2, \; ID_{FCI}=2$, the FCI correctly identifies the ID of the dataset;
 (\subref{fig:FCIexpl:Model_dataset_swissroll_3d}) A dataset with points distributed on a Swiss Roll $ID=2, \; ID_{FCI}=3$,
  the FCI fails to identify the ID of the dataset due to the curvature of the manifold.}\label{fig:FCIexpl}
 \end{figure}


\begin{figure}
 % Local FCI and our way to use the methodology
 \centering
 \begin{subfigure}[b]{\textwidth}
   \centering
   \includegraphics[width=\textwidth]{Local_FCI/localFCI.pdf}
   \subcaption{}\label{fig:Local_FCI:localFCI}
 \end{subfigure}\\
 \begin{subfigure}[b]{0.45\textwidth}
   \centering
   \includegraphics[width=\textwidth]{Model_dataset/hist2d_plane_localFCI.pdf}
   \subcaption{}\label{fig:Local_FCI:2dhist_plane}
 \end{subfigure}
 \begin{subfigure}[b]{0.45\textwidth}
   \centering
   \includegraphics[width=\textwidth]{Model_dataset/hist2d_swissroll_localFCI.pdf}
   \subcaption{}\label{fig:Local_FCI:2dhist_swissroll}
 \end{subfigure}
 \caption{Graphical explanation of the Local version of FCI (Local FCI). As a model dataset we used a swiss roll;
 on the first row in (\subref{fig:Local_FCI:localFCI}) we show the neighborhoods of different sizes used to compute the FCI;
 on the second row we show the $D-1$ sphere of the normalized neighborhoods with highlighted some sphere of radius $r$;
 To compute the ID using the Local FCI method we propose to plot the distribution of IDs as a function of the ratio $\delta =d_b/<d>$,
 where $d_b$ is the minimal distance between the barycenter of the neighborhood points and the datapoints and $<d>$ is the average distance between the points in the dataset.
 (\subref{fig:Local_FCI:2dhist_plane}) IDs distribution for a plane; (\subref{fig:Local_FCI:2dhist_swissroll}) IDs distribution for a swiss roll.
The position of the peak in the histogram is the ID of the dataset. Local FCI correctly estimates the ID of the dataset as 2 for both datasets.}\label{fig:Local_FCI}
\end{figure}

\begin{figure}
% RNN task: network structure, input and output activity
 \centering
 \begin{subfigure}[b]{0.80\textwidth}
   \centering
   \includegraphics[width=\textwidth]{RNN_task_annotated_paper.pdf}
   \subcaption{}\label{fig:RNN_task:RNN_structure}
 \end{subfigure}\\
 \begin{subfigure}[b]{0.29\textwidth}
   \centering
   \includegraphics[width=\textwidth]{reactgo_inout_activity.pdf}
   \subcaption{}\label{fig:RNN_task:reactgo}
 \end{subfigure}
 \begin{subfigure}[b]{0.29\textwidth}
   \centering
   \includegraphics[width=\textwidth]{fdgo_inout_activity.pdf}
   \subcaption{}\label{fig:RNN_task:fdgo}
 \end{subfigure}
\begin{subfigure}[b]{0.29\textwidth}
   \centering
   \includegraphics[width=\textwidth]{contextdm1_inout_activity.pdf}
   \subcaption{}\label{fig:RNN_task:contextdm1}
\end{subfigure}
\begin{subfigure}[b]{0.10\textwidth}
   \centering
   \includegraphics[width=\textwidth]{colorbar.pdf}
   \subcaption{}\label{fig:RNN_task:colorbar}
\end{subfigure}
\caption{
  \subref{fig:RNN_task:RNN_structure} Schematic structure of the RNN used for the task.
  The network has $65$ input units: one for the fixation $I_{fix}$;
   and $64$ encoding for the angles in the perceptual task divided into two set of $32$ channels each $I_1,\; I_2$. 
  In the recurrent part it has $256$ neurons and the output layer has $33$ units: one for the fixation, $O_{fix}$,
  and $32$ units for the angles in the perceptual task $O$.
  \subref{fig:RNN_task:reactgo} Input, output and network activity for the Reactgo task;
  \subref{fig:RNN_task:fdgo} Input, output and network activity for the Fdgo task;
  \subref{fig:RNN_task:contextdm1} Input, output and network activity for the Contextdm1 task.
  \subref{fig:RNN_task:colorbar} Colorbar for the previous plots.
  $N$ activity of the recurrent part of the trained network;
  $I_1,\;I_2$ activity in the input layer;
  $O$ activity in the output layer.
  The green vertical lines are placed at the end of a trial.
}\label{fig:RNN_task}
\end{figure}


\begin{figure}
 % Reactgo
 \centering
 \begin{subfigure}[b]{0.25\textwidth}
   \centering
   \includegraphics[width=\textwidth]{Reactgo/exp_var_reactgo.pdf}
   \subcaption{}\label{fig:Reactgo:exp_var_reactgo}
 \end{subfigure}
 \begin{subfigure}[b]{0.30\textwidth}
   \centering
   \includegraphics[width=\textwidth]{Reactgo/reactgo_3d.pdf}
   \subcaption{}\label{fig:Reactgo:reactgo_3d}
 \end{subfigure}
 \begin{subfigure}[b]{0.40\textwidth}
   \centering
   \includegraphics[width=\textwidth]{Reactgo/reactgo_3d_2.pdf}
   \subcaption{}\label{fig:Reactgo:reactgo_3d_2}
 \end{subfigure}\\
  \begin{subfigure}[b]{0.45\textwidth}
    \centering
    \includegraphics[width=\textwidth]{reactgo_localID.pdf}
   \subcaption{}\label{fig:Reactgo:hist2D_reactgo_localFCI}
  \end{subfigure}
 \begin{subfigure}[b]{0.45\textwidth}
    \centering
    \includegraphics[width=\textwidth]{Reactgo/lle_2d_reactgo.pdf}
   \subcaption{}\label{fig:Reactgo:lle_2d_reactgo}
  \end{subfigure}
 \caption{Reactgo task
  (\subref{fig:Reactgo:exp_var_reactgo}) Explained variance of the different PCs of the network activity.
  Looking at the variance, PCA predicts a dimensionality of 1 if we consider the highest drop in  variance  or a dimension of 7 if we consider the second shoulder in variance. 
  (\subref{fig:Reactgo:reactgo_3d}) Projection of the network activity on the first three PCs. The colorcode stands for the different input angles of the trials.
   Only the final part of the trial (when the network produces an output) is shown.
  To correctly encode the angle the network needs two components ($PC_2$ and $PC_2$) while $PC_1$ represents the time evolution.
  (\subref{fig:Reactgo:reactgo_3d_2}) shows that the ``cone'' representing the response is distributed in different PCs.
  (\subref{fig:Reactgo:hist2D_reactgo_localFCI}) The Local FCI for the reactgo task predicts an intrinsic dimension $ID=2.05$;
  (\subref{fig:Reactgo:lle_2d_reactgo}) The LLE for the reactgo task shows an activity distributed on a circular structure. 
  Each trial is a sequence of points with the same color.
   }\label{fig:Reactgo}
\end{figure}


\begin{figure}
 % Fdgo
 \centering
 \begin{subfigure}[b]{0.20\textwidth}
   \centering
   \includegraphics[width=\textwidth]{Fdgo/exp_var_fdgo.pdf}
  \subcaption{}\label{fig:Fdgo:exp_var_fdgo}
 \end{subfigure}
 \begin{subfigure}[b]{0.23\textwidth}
   \centering
   \includegraphics[width=\textwidth]{Fdgo/fdgo_3d.pdf}
  \subcaption{}\label{fig:Fdgo:fdgo_3d}
 \end{subfigure}
 \begin{subfigure}[b]{0.18\textwidth}
   \centering
   \includegraphics[width=\textwidth]{Fdgo/fdgo_3d_highlight.pdf}
  \subcaption{}\label{fig:Fdgo:fdgo_3d_highlight}
 \end{subfigure}
 \begin{subfigure}[b]{0.35\textwidth}
   \centering
   \includegraphics[width=\textwidth]{Fdgo/fdgo_3d_temporal_highlight.pdf}
  \subcaption{}\label{fig:Fdgo:fdgo_3d_temporal}
 \end{subfigure}\\
  \begin{subfigure}[b]{0.45\textwidth}
    \centering
    \includegraphics[width=\textwidth]{fdgo_localID.pdf}
  \subcaption{}\label{fig:Fdgo:hist2D_fdgo_localFCI}
  \end{subfigure}
 \begin{subfigure}[b]{0.45\textwidth}
    \centering
    \includegraphics[width=\textwidth]{Fdgo/lle_fdgo.pdf}
  \subcaption{}\label{fig:Fdgo:lle_2d_fdgo}
  \end{subfigure}
 \caption{Fdgo task
(\subref{fig:Fdgo:exp_var_fdgo}) Explained variance of the different PCs of the network activity.
Looking at the variance, there is not a clear drop in variance that helps in identifying the ID of the dataset.
(\subref{fig:Fdgo:fdgo_3d}) Projection of the network activity on the first three PCs. The colorcode stands for the different input angles of the trials.
Looking at the projection on the first three PCs seems that the network has activity has a crossing.
(\subref{fig:Fdgo:fdgo_3d_highlight}) and (\subref{fig:Fdgo:fdgo_3d_temporal}) Considering only the points at a fixed time appears that the crossing is only an artifact due to the projection;
(\subref{fig:Fdgo:hist2D_fdgo_localFCI}) The Local FCI for the fdgo task predicts an intrinsic dimension $ID=2.05$;
(\subref{fig:Fdgo:lle_2d_fdgo}) The LLE for the fdgo task shows an activity distributed on a circular structure.
 Each trial is a sequence of points with the same color.
 }\label{fig:Fdgo}
\end{figure}


\begin{figure}
 % Contextdm1
 \centering
 \begin{subfigure}[b]{0.25\textwidth}
   \centering
   \includegraphics[width=\textwidth]{Contextdm1/exp_var_contextdm1.pdf}
  \subcaption{}\label{fig:Contextdm1:exp_var_contextdm1}
 \end{subfigure}
 \begin{subfigure}[b]{0.30\textwidth}
   \centering
   \includegraphics[width=\textwidth]{Contextdm1/contextdm1_3d.pdf}
  \subcaption{}\label{fig:Contextdm1:contextdm1_3d}
 \end{subfigure}
 \begin{subfigure}[b]{0.40\textwidth}
   \centering
   \includegraphics[width=\textwidth]{Contextdm1/contextdm1_3d_2.pdf}
  \subcaption{}\label{fig:Contextdm1:contextdm1_3d_2}
 \end{subfigure}\\
  \begin{subfigure}[b]{0.45\textwidth}
    \centering
    \includegraphics[width=\textwidth]{contextdm1_localID.pdf}
  \subcaption{}\label{fig:Contextdm1:hist2D_contextdm1_localFCI}
  \end{subfigure}
 \begin{subfigure}[b]{0.45\textwidth}
    \centering
    \includegraphics[width=\textwidth]{Contextdm1/lle_2d_contextdm1.pdf}
  \subcaption{}\label{fig:Contextdm1:lle_2d_contextdm1}
  \end{subfigure}
 \caption{Contextdm1 task
(\subref{fig:Contextdm1:exp_var_contextdm1}) Explained variance of the different PCs of the network activity.
Looking at the variance, there is not a clear drop in variance that helps in identifying the ID of the dataset.
(\subref{fig:Contextdm1:contextdm1_3d}) Projection of the network activity on the first three PCs. The colorcode stands for the different input angles of the trials.
Looking at the projection on the first three PCs it is ont clear the encoding of the angle;
(\subref{fig:Contextdm1:contextdm1_3d_2}) shows how the other two PCs $PC_4$ and $PC_5$  adds some information but are not detrimental;
(\subref{fig:Contextdm1:hist2D_contextdm1_localFCI}) The Local FCI for the contextdm1 task predicts an intrinsic dimension $ID=5.16$;
(\subref{fig:Contextdm1:lle_2d_contextdm1}) The LLE for the contextdm1 task shows an activity distributed on an intricate structure.
 }\label{fig:Contextdm1}
\end{figure}



\begin{figure}
    \centering
    \includegraphics[width=0.9\textwidth]{local_FCI_violin.pdf}
    \caption{Local FCI for all the tasks.
    The ID plot shows that all the tasks are low dimensional with $ID<6$ and most of them have 
    $ID\sim 2$ showing that a circle is enough to represent the angle. 
    }\label{fig:local_FCI_alltasks}
\end{figure}

\begin{figure}
    \centering
    \begin{subfigure}[b]{0.45\textwidth}
      \centering
      \includegraphics[width=\textwidth]{PCA_95_violin.pdf}
      \subcaption{}\label{fig:PCA_ID_alltasks:PCA}
    \end{subfigure}
    \begin{subfigure}[b]{0.45\textwidth}
      \centering
      \includegraphics[width=\textwidth]{PCA_PR_violin.pdf}
      \subcaption{}\label{fig:PCA_ID_alltasks:PR}
    \end{subfigure}
    \caption{ID with PCA for all the tasks.
    (\subref{fig:PCA_ID_alltasks:PCA} )  ID as the number of PCs needed to explain the $95\%$ of the variance.
      PCA predict an ID that is larger than the one predicted by Local FCI for the same task
      (\subref{fig:PCA_ID_alltasks:PR} ) ID as the Participation Ratio for the different tasks.
      For some families of tasks it fails to estimate the right ID.
       e.g. For the react family it estimate an ID of 1 but the true ID is 2
        }\label{fig:PCA_ID_alltasks}
\end{figure}
\end{document}